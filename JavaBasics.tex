\documentclass[11pt]{beamer}

% list of packages used
\usepackage[latin1]{inputenc}
\usepackage{amsmath}
\usepackage{amsfonts}
\usepackage{amssymb}
\usepackage{graphicx}
\usepackage{listings}

% Author details
\author{Vinay @EduTechLabs}
\title{Java Basics}

% configs

%%set theme 
\usetheme{Warsaw}

%%configure listings package for java
% Default fixed font does not support bold face
\DeclareFixedFont{\ttb}{T1}{txtt}{bx}{n}{8pt} % for bold
\DeclareFixedFont{\ttm}{T1}{txtt}{m}{n}{8pt}  % for normal

%custom colours
\definecolor{pblue}{rgb}{0.13,0.13,1}
\definecolor{pgreen}{rgb}{0,0.5,0}
\definecolor{pred}{rgb}{0.9,0,0}
\definecolor{pgrey}{rgb}{0.46,0.45,0.48}

\definecolor{javared}{rgb}{0.6,0,0} % for strings
\definecolor{javagreen}{rgb}{0.25,0.5,0.35} % comments
\definecolor{javapurple}{rgb}{0.5,0,0.35} % keywords
\definecolor{javadocblue}{rgb}{0.25,0.35,0.75} % javadoc

%configuring listings package to highlight java syntax
\newcommand\javastyle{\lstset{
language=Java,
basicstyle=\ttfamily,
keywordstyle=\color{javapurple}\bfseries,
stringstyle=\color{javared},
commentstyle=\color{javagreen},
morecomment=[s][\color{javadocblue}]{/**}{*/},
numbers=left,
numberstyle=\tiny\color{black},
stepnumber=1,
tabsize=4,
showspaces=false,
otherkeywords={this, class, enum},             % Add keywords here
%keywordstyle=\ttb\color{deepblue},
emph={MyClass,Pizza,className},          % Custom highlighting
emphstyle=\ttb\color{pred},    % Custom highlighting style
moredelim=[il][\textcolor{pgrey}]{$$}, %$$ %the double dollar to prevent texmaker from going crazy
%moredelim=[is][\textcolor{pgrey}]{\%\%}{\%\%}
%frame=tb,                         % Any extra options here
showstringspaces=false, 
}}

% java environment
\lstnewenvironment{java}[1][]
{
\javastyle
\lstset{#1}
}
{}

% java for external files
\newcommand\javaexternal[2][]{{
\javastyle
\lstinputlisting[#1]{#2}}}

% java for inline
\newcommand\javainline[1]{{\javastyle\lstinline!#1!}}

%%%%%%%%%%%% begin writing the actual document %%%%%%%%%%%%%%%%%
\begin{document}

%get the title page
\frame{\maketitle}
% welcome page 
\begin{frame}{Welcome to the course}
	\begin{center}
	 Welcome to \emph{\color{pblue}EduTechLab's} basic java Course.
	\end{center}
\end{frame}

%%%%%%%%%%%%%%% about java

% preparing the code
\defverbatim[colored]\pycode{%
\begin{java}

\end{java}
}
%%%%%%%%%%%%%%%%%%%%

%START MAKING SLIDES FROM HERE ON

%%%%%%%%%%%%%%%%%%%%

\begin{frame}[containsverbatim]{Hello World}
\begin{java}
public class MyClass{
/* This will program print 
 * `Hello World' as the output
 */
	public static void main(String[]args){
		System.out.println("Hello World"); 
		// prints Hello World
	}
}
\end{java}
\end{frame}

\begin{frame}[containsverbatim]{Compile and run java program}
	\begin{flushleft}
	
	
		\item{Running a java program} \\
		\begin{java}
		javac HelloWorld.java
		java HelloWorld
		\end{java}
		\item{Note: navigate to java file location through terminal before executing above commands.}
	\end{flushleft}
	
\end{frame}

\begin{frame}[containsverbatim]{Java Comments}
\begin{java}
public class HelloWorld{

public static void main(String[]args){
  // an example of single line comment
  System.out.println("Hello World");
  /* Multi line commants 
   are given like this. */
  }
}
\end{java}
\end{frame}

\begin{frame}[containsverbatim]{Java Class Structure}
\begin{java}
public class ClassName{
  String name;      // instance variable
  static int count; // class variable

  public ClassName(String name){ // constructor
    this.name = name;  
  }

  void method1(){
    String address; // local variable
  }  
}
\end{java}

\end{frame}


\begin{frame}[containsverbatim]{Enum Example}
\begin{java}
class Pizza{
	enum PizzaSize{ SMALL, MEDUIM, LARGE }
	PizzaSize size;
}

public class PizzaTest{
	public static void main(String args[]){
		Pizza p =new Pizza();
		p.size =Pizza.PizzaSize.MEDUIM ;
	}
}
\end{java}
\end{frame}


\begin{frame}[containsverbatim]{Basic Class Syntax}
\begin{java}
public class Dog{
	String breed;
	int age;
	String color;

	void barking(){
	}

	void hungry(){
	}

	void sleeping(){
	}
}
\end{java}
\end{frame}


\begin{frame}[containsverbatim]{Constructors}
\begin{java}
public class Person{

	String name;
	
	public Person(){
	}
	
	public Person(String name){
		this.name = name;
	}
}
\end{java}
\end{frame}

\begin{frame}[containsverbatim]{Java Constructor}
\begin{java}
public class Dog{
  String name;  
  
  public Dog(){
  }

  public Dog(String name){
  this.name = name;
  }
}
\end{java}


\begin{java}
Dog Bruno = new Dog();
Dog Bruno = new Dog("Bruno");
\end{java}

\end{frame}

\begin{frame}[containsverbatim]{Getters and Setters}
\begin{java}
public class Dog{
  String name;  
  
  public Dog(){
  }
  public Dog(String name){
    this.name = name;
  }
  
  public String getName(){
    return this.name;  
  }
  public void setName(String name){
    this.name = name;
  }  
}
\end{java}
\end{frame}


\begin{frame}[containsverbatim,allowframebreaks]{Things to remember while creating Classes}

\begin{enumerate}
\item One .java file can have multiple classes.
\item Only one class per .java file can be public.
\item All remaining classes must be non-public.
\item Name of the public class must be the name of .java class.
\item class declared in package must begin with the package identification line
\begin{java}
package com.example;
\end{java}


\item import statements must be written between package declaration and class definition
\begin{java}
package com.example;
import ...
public class Dog{
  ....
}
\end{java}
\end{enumerate}
\end{frame}

\begin{frame}[containsverbatim,allowframebreaks]{Data Types}
\begin{enumerate}
\item Primitive Data Types
\begin{enumerate}
\item byte
\item short
\item int
\item long
\item float
\item double
\item boolean
\item char
\end{enumerate}
\item Object Data Types
\begin{enumerate}
\item Dog
\item Person ...
\end{enumerate}
\end{enumerate}
\end{frame}


\begin{frame}[containsverbatim]{Inheretence -- IS -- A Relation}
\begin{java}
public class Animal{
	.....
}

  public class Mammal extends Animal{
  	.....
  }

    public class Dog extends Mammal{
    		.....
    }

  public class Reptile extends Animal{
  	.....
  }
\end{java}
\end{frame}


\begin{frame}[containsverbatim]{IS -- A vs HAS -- A Relation}
\begin{java}

public class Vehicle{
  ....
}

public class Engine{
  ....
}

public class Car extends Vehicle{//IS-A relation
  private Engine e;  //HAS-A relation
  ....
}
\end{java}
\end{frame}



\begin{frame}[containsverbatim]{Singleton Class}
\begin{java}
public class SingletonClass{
private static SingletonClass instance = null;
	
	protected SingletonClass(){
		// Exists only to defeat instantiation.
	}
	public static SingletonClass getInstance(){
		if(instance == null){
			instance = new SingletonClass();
		}
		return instance;
	}
}
\end{java}
\end{frame}

\begin{frame}[containsverbatim]{Creating Objects}
\begin{java}
import package.name.Person;

public class PersonTest{

	public static void main(String[]args){
	
		// created the reference which 
		// will point to object
		Person ram;
		
		// assign an object to reference
		ram = new Person("Ram");
	}
}
\end{java}
\end{frame}

\begin{frame}[containsverbatim]{Creating Objects}
\begin{java}

\end{java}
\end{frame}


\begin{frame}[containsverbatim]{Polymorphism}

\end{frame}



\begin{frame}[containsverbatim,allowframebreaks]{Method Overloading}

\end{frame}



\begin{frame}[containsverbatim,allowframebreaks]{Constructor Overloading}

\end{frame}



\begin{frame}[containsverbatim,allowframebreaks]{Collections}

\end{frame}



\begin{frame}[containsverbatim,allowframebreaks]{Arraylist}

\end{frame}



\begin{frame}[containsverbatim,allowframebreaks]{Iterator}

\end{frame}


\begin{frame}[containsverbatim,allowframebreaks]{Loops}

\end{frame}



\begin{frame}[containsverbatim,allowframebreaks]{for loop}

\end{frame}



\begin{frame}[containsverbatim,allowframebreaks]{while loop}

\end{frame}



\begin{frame}[containsverbatim,allowframebreaks]{do while}

\end{frame}



\begin{frame}[containsverbatim,allowframebreaks]{for loop iterator}

\end{frame}



\begin{frame}[containsverbatim,allowframebreaks]{typecasting}

\end{frame}



\begin{frame}[containsverbatim,allowframebreaks]{Access Modifiers -- Default, Public, Private, Protected}

\begin{enumerate}
\item \emph{Default} -- Visible to the package,  No modifiers are needed.
\item \emph{Private} -- Visible to the class only.
\item \emph{Public} -- Visible to the world.
\item \emph{Protected} -- Visible to the package and all subclasses.    
\end{enumerate}






\end{frame}



\begin{frame}[containsverbatim,allowframebreaks]{static, final, abstract}

\end{frame}



\begin{frame}[containsverbatim]{try catch finally}

\end{frame}


\begin{frame}[containsverbatim,allowframebreaks]{if else}

\end{frame}



\begin{frame}[containsverbatim,allowframebreaks]{switch}

\end{frame}


\begin{frame}[containsverbatim,allowframebreaks]{Interfaces}

\end{frame}


\begin{frame}[containsverbatim,allowframebreaks]{dconstructor}

\end{frame}



\begin{frame}[containsverbatim,allowframebreaks]{utility functions -- parseint etc.}

\end{frame}



\begin{frame}[containsverbatim,allowframebreaks]{threads}

\end{frame}


\begin{frame}[containsverbatim,allowframebreaks]{Inner class}

\end{frame}


\begin{frame}[containsverbatim,allowframebreaks]{scope}

\end{frame}


\begin{frame}[containsverbatim,allowframebreaks]{packages}

\end{frame}


\begin{frame}[containsverbatim,allowframebreaks]{HashMap}

\end{frame}


\begin{frame}[containsverbatim,allowframebreaks]{stack memory}

\end{frame}


\begin{frame}[containsverbatim,allowframebreaks]{static memory}

\end{frame}


\begin{frame}[containsverbatim,allowframebreaks]{heap memory -- object location}

\end{frame}


\begin{frame}[containsverbatim,allowframebreaks]{Array of Objects}

\end{frame}




\end{document}